\documentclass[programming]{../../../../mfcs}
\newcommand{\projectduedate}{Tuesday, May 2nd}
\newcommand{\quarter}{Spring 2017}
\usepackage{color}
\usepackage{multicol}

\course{CSE 154}{Web Programming}{\quarter}
\NoDate
\topic{Homework Assignment 4: Fifteen Puzzle}
\usepackage{csquotes}
\usepackage{tabularx}

\begin{document}
\vspace{-3.8em}

\hfill\begin{varwidth}{0.5\textwidth}
{\large {\bf\color{colour} Due Date:} \projectduedate}
\end{varwidth}
\vspace{2em}

This assignment is about JavaScript's Document Object Model (DOM) and events. You'll write an
interactive
"Fifteen Puzzle" page, following the specifications provided below. The example output of this page
is provided in the resources folder on the course calendar.

\vspace{1em}
\begin{question}{Background Information}
The Fifteen Puzzle (also called the Sliding Puzzle) is a simple classic game consisting of a 4x4 grid of numbered
squares with one square missing. The object of the game is to arrange the tiles into numerical order by repeatedly
sliding a square that neighbors the missing square into its empty space.
\newline

  You will write the \texttt{CSS} and JavaScript code for a page \texttt{\textbf{\color{colour}{fifteen.html}}} that plays the Fifteen Puzzle. You will also submit
a background image of your own choosing, displayed underneath the tiles of the board. Choose any image you
like, so long as its tiles can be distinguished on the board. Turn in the following files:
\begin{itemize}
  \item \texttt{\textbf{\color{colour}{fifteen.js}}} the JavaScript code for your web page
  \item \texttt{\textbf{\color{colour}{fifteen.css}}} the \texttt{CSS} styles for your web page
  \item \texttt{\textbf{\color{colour}{background.jpg}}} the background image, suitable for a puzzle of size 400x400 (go find any image you like, such as by Google Image Search, but don't
      use the Mario image in the output)
\end{itemize}

  You will not submit any \texttt{.html} file, nor directly write any \texttt{HTML} code. We will
  provide you with the \texttt{HTML} code to
  use, which should not be modified. (Download the \texttt{.html} file to your machine while writing your JavaScript code,
but your code should work with the provided files unmodified.) You will write JavaScript code that interacts with the
page using the DOM. To modify the page's appearance, write appropriate DOM code to change styles of on-screen
elements by setting classes, IDs, and/or style properties on them.
\newline

  \textbf{\color{colour}{Tip for playing the game}}: First get the entire top/left sides into proper position. That is, put squares number 1, 2, 3, 4, 5, 9, and 13
into their proper places. Now never touch those squares again. Now what's left to be solved is a 3x3 board, which is much easier.
\end{question}
\begin{question}{Appearance Details}
All text on the page is displayed in a "cursive" font family, at a default font size of 14pt. Everything on the page is
centered, including the top heading, paragraphs, the puzzle, the Shuffle button, and the W3C buttons at bottom.
\newline

In the center of the page are fifteen tiles representing the puzzle. The overall puzzle occupies 400x400 pixels on
the page, horizontally centered. Each puzzle tile occupies a total of 100x100 pixels, but 5px on all four sides are
  occupied by a black border. This leaves 90x90 pixels of area inside each tile.\newline

  The \texttt{HTML} file given to you does not
  contain the fifteen div elements to represent the puzzle pieces, \textbf{and you are not supposed to modify
  the \texttt{HTML} file};
so you will have to create the puzzle pieces and add them to the page yourself using the JavaScript DOM. Before the
  \texttt{load} event fires on the window, the page should not contain any puzzle pieces. Once the window loads,
the page should appear with the puzzle in its properly arranged order like in the screenshot on the first page of this
spec, with 1 at top-left, 4 at top-right, 13 at bottom-left, the empty square at bottom-right, and so on.
Each tile displays a number from 1 to 15, in a 40pt font. Each tile displays part of the image
  \texttt{background.jpg}, which
you should put in the same folder as your page. Which part of the image is displayed by each tile is related to that
tile's number. The "1" tile shows the top-left 100x100 portion of the image. The "2" tile shows the next 100x100px
of the background that would be to the right of the part shown under the "1" tile, and so on.
\newline

Your background image appears on the puzzle pieces when you set it as the background-image of each
  piece. By adjusting the \texttt{background-position} of each div, you can show a different part of the background on each
  piece. One confusing thing about \texttt{background-position} is that the x/y values shift the background behind the
element, not the element itself. The offsets are the negation of what you may expect. For example, if you wanted a
100x100px div to show the top-right corner of a 400x400px image, set its background-position property to 300px 0px . 
\newline

The following is a complete listing of the exact background-position values each location on the
board should have:
\newline

  \begin{center}
  \begin{tabular}{|l|l|l|l|}
    \hline
  0px 0px  & -100px 0px  & -200px 0px  & -300px 0px
 \\\hline 
  0px -100px & -100px -100px & -200px -100px  & -300px -100px
 \\\hline 

  0px -200px  & -100px -200px &  -200px -200px  & -300px -200px
 \\\hline 

    0px -300px & -100px -300px &  -200px -300px &  
  \\\hline
  \end{tabular}
\end{center}

  \vspace{0.5em}
Centered under the puzzle tiles is a Shuffle button that can be clicked to randomly rearrange the tiles of the puzzle.
See the "Shuffle Algorithm" section of this spec for more details about implementing the shuffle behavior.
\newline

All other style elements on the page are subject to the preference of the web browser. The screenshots in this
document were taken on Windows in Firefox, and cursive fonts vary quite a bit from system to system, so it is
likely that a screenshot of your page will look somewhat different than the expected output.
\end{question}
\begin{question}{Playing the Game}
When the mouse button is pressed on a puzzle square, if that square is next to the blank square, it is moved into the
blank space. If the square does not neighbor the blank square, no action occurs. Similarly, if the mouse is pressed
on the empty square or elsewhere on the page, no action occurs.
\newline

When the mouse hovers over a square that can be moved (neighbors the blank spot), its border and text color
should become red. Also, the mouse cursor should change into a "hand" cursor pointing at the square (do this by
setting the \texttt{CSS} cursor property to pointer.) Once the cursor is no longer hovering on the square, its appearance
should revert to its original state. When the mouse cursor hovers over a square that cannot be moved, it should use
the system's standard arrow cursor (set the cursor property to default) and it should not have the red text or
  borders or other effects. (You may find it useful to use the \texttt{:hover CSS} pseudo-class to help deal with hovering.)
\newline

The game is not required to take any particular action when the puzzle has been won. You can decide if you'd like
to pop up an alert box congratulating the user or add any other optional behavior to handle this event.
\newline

\subquestion*{Shuffle Algorithm}
Centered under the puzzle tiles is a Shuffle button that can be clicked to randomly rearrange the tiles of the puzzle.
When the Shuffle button is clicked, the puzzle tiles are rearranged into a random ordering so that the user has a
challenging puzzle to solve.
\newline

The tiles must be rearranged into a solvable state. Some puzzle states are not solvable; for example, the puzzle
cannot be solved if you swap only its 14 and 15 tiles. Therefore your algorithm for shuffling cannot simply move
each tile to a completely random location. It must generate only solvable puzzle states if you want full credit.
\newpage

We suggest that you generate a random valid solvable puzzle state by repeatedly choosing a random neighbor of the
missing tile and sliding it onto the missing tile's space. Roughly 1000 such random movements should produce a
well-shuffled board. Here is a rough pseudo-code of the algorithm we suggest for shuffling.

\begin{lstlisting}
for (~1000 times):
    neighbors = []
    for each neighbor n that is directly up, down, left, right from empty square:
        if n exists and is movable:
            neighbors.push(n)
    randomly choose an element i from neighbors
    move neighbors[i] to the location of the empty square
\end{lstlisting}

Notice that on each pass of our algorithm, it is guaranteed that one square will move. Some students write an
algorithm that randomly chooses any one of the 15 squares and tries to move it; but this is a poor way to shuffle
because many of the 15 squares are not neighbors of the empty square. Therefore the loop must repeat many more
times in order to shuffle the elements effectively, making it slow and causing the page to lag. This is not acceptable.
\newline

\textbf{Your algorithm should be efficient}; if it takes more than 1 second to run or performs a large number of tests and
calls unnecessarily, you may lose points. For full credit, your shuffle should have a good random distribution and
thoroughly rearrange the tiles and the blank position. You are not required to follow exactly the algorithm above,
but if you don't, your algorithm must exhibit the desired properties described in this section to receive full credit.
\newline

Your shuffle algorithm will need to incorporate randomness. You can generate a random integer from 0 to K,
  which is helpful to randomly choose between K choices, by writing, \texttt{parseInt(Math.random() * K)}.
\newline

We suggest first implementing code to perform a single random move; that is, when Shuffle is clicked, randomly pick
one square near the empty square and move it. Get it to do this once then work on doing it many times in a loop.
\newline

In your shuffle algorithm, or elsewhere in your program, you may want access to the DOM object for a square at a
particular row/column or x/y position. We suggest you write a function that accepts a row/column as parameters
and returns the DOM object for the corresponding square. It may be helpful to you to give an id to each square,
such as "square\_2\_3" for the square in row 2, column 3, so that you can more easily access the squares later in
your JavaScript code. (If any square moves, you will need to update its id value to match its new location.)
Use these ids appropriately. Don't break apart the string "square\_2\_3" to extract the 2 or 3. Instead, use the ids
to access the corresponding DOM object: e.g., for the square at row 2, column 3, make an id string of
"square\_2\_3".
\end{question}

\begin{question}{Development Strategy}
Students generally find this to be a tricky assignment. Here is a suggested ordering for tackling the steps:
\begin{enumerate}[{1.}]
\item Make the fifteen puzzle pieces appear in the correct positions without any background behind them.
\item Make the correct parts of the background show through behind each tile.
\item Write the code that moves a tile when it is clicked from its current location to the empty square's location.
Don't worry initially about whether the clicked tile is "movable" (whether it is next to the empty square).
\item Write code to determine whether a square can move or not (whether it neighbors the empty square).
\item Implement a highlight when the mouse hovers over movable tiles. Track where the empty square is at all times.
\end{enumerate}

\subquestion*{Hints}
\begin{itemize}
\item Use absolute positioning to set the x/y locations of each puzzle piece. The overall puzzle area must use a
relative position in order for the x/y offsets of each piece to be relative to the puzzle area's location.

\item You can convert a string to a number using \texttt{parseInt}. This also works for strings that start with a number and
end with non-numeric content. For example, \texttt{parseInt("123four")} returns 123.

\item Many students have bugs related to not setting their DOM style properties using proper units and formatting. The string you assign in your \texttt{JS} code must exactly match what would have been in the\texttt{CSS} file. For example, if
you want to set the size of an element, a value like 42 or "42" will fail, but "42px" or "42em" will succeed.
When setting a background image, a value like "\texttt{foo.jpg}" will fail, but
"\texttt{url(foo.jpg)}" will succeed. When
setting a background position, "42 35" will fail but "42px 35px" will succeed. And so on.

\item We suggest that you do not explicitly make a div to represent the empty square of the puzzle. Keep track of
where it is, such as by row/column or by x/y position, but don't create an actual DOM element for it. We also
suggest that you not store your puzzle squares in a 2-D array. This might seem like a good structure because of
the 4x4 appearance of the grid, but will likely make it more difficult to keep it up to date as the squares move.
Furthermore, it is easy to write 2-D array code that ends up being poor style because of how difficult it
is to maintain the proper state of the arrays.

\item Many students have redundant code because they don't create helper functions. You should write functions
for common operations, such as moving a particular square, or for determining whether a given square currently
can be moved. The this keyword (see book section 9.1) can be helpful for reducing redundancy.
\end{itemize}
\end{question}

\begin{question}{Implementation and Grading}
For full credit, your \texttt{HTML} and \texttt{CSS} code must pass the W3C validators. Your\texttt{HTML} and\texttt{CSS} should also follow
the style guide posted on the class web site.
\newline

Separate content (\texttt{HTML}), presentation (\texttt{CSS}), and behavior (\texttt{JS}).
Your \texttt{JS} code should use styles and classes from the\texttt{CSS} rather than manually
setting each style property in the \texttt{JS}. For example, rather than setting the
\texttt{.style} of a
DOM object, instead, give it a \texttt{className} and put the styles for that class in your\texttt{CSS} file. Style properties related
to x/y positions of tiles and their backgrounds are impractical to put in the\texttt{CSS} file, so those can be in your \texttt{JS} code.
Use unobtrusive JavaScript so that no JavaScript code, onclick handlers, etc. are embedded into the\texttt{HTML} code.
You should not use any external JavaScript frameworks or libraries such as jQuery to solve this assignment.
\newline

Your JavaScript code should pass our JSLint tool with no errors.
\newline

Your\texttt{.js} file must run in strict mode by putting "use strict"; within your module.
Follow the JavaScript style guidelines posted on the class web site. Make extra effort to minimize redundant code.
Capture common operations as functions to keep code size and complexity from growing. You can reduce your
      code size by using the \texttt{this} keyword in your event handlers.
\newline

No global variables or functions are allowed. To avoid globals, use the module pattern as taught in lecture,
wrapping your code in an anonymous function invocation. Even if you use the module pattern, limit the amount of
"module-global" variables to those that are truly necessary; values should be local as much as possible. If a particular
      literal value is used frequently, declare it as a module-global "constant" variable
      \texttt{IN\_UPPER\_CASE} and use the
constant in your code.
\newline

Do not store DOM element objects, such as those returned by
\texttt{document.getElementById} or \texttt{document.querySelectorAll}, as module-global variables. As a reference, our
own solution has only four module-global variables: the empty square's row and column (initially both are set to 3),
the number of rows/cols in the puzzle (4), and the pixel width/height of each tile (100). The last two are essentially
constants.
\newline

Your JavaScript code and \texttt{CSS} files should have adequate commenting. The top of your
\texttt{JS} file should have a descriptive
comment header describing the assignment, and each function and complex section of code should be documented.
Format your code similarly to the examples from class. Properly use whitespace and indentation. Use good
variable and method names. Avoid lines of code more than 100 characters wide. For reference, our \texttt{.js} file has
roughly 160 lines (110 "substantive"), and our \texttt{CSS} file has roughly 50 lines.
\newline

Do not place your solution on a public web site. Submit your own work and follow the course misconduct policy.
\end{question}
\end{document}
